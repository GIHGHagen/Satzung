\pagestyle{fancy}
\begin{contract}
    \Clause{title=Name und Sitz}
\label{sec:name}

Die studentische Initiative führt den Namen ``Gesellschaft für Informatik Hochschulgruppe Hagen'' (GI HSG).  Sie hat ihren Sitz in Hagen.


\Clause{title=Zweck der Initiative}
\label{sec:zweck}

Der satzungsmäßige Zweck der Initiative ist die Förderung des Fachbereichs Informatik an der Hochschule sowie die Unterstützung und Weiterentwicklung des Studiums, der Forschung und des Austauschs von Wissen im Bereich der Informatik. 
Die Hochschulgruppe strebt die Vernetzung von Studierenden, Lehrenden und Fachleuten der Informatik an, um den Dialog, die Kooperation und die Zusammenarbeit in diesem Fachbereich inneruniversitär als auch überregional zu fördern.
Auch richten wir uns an Auszubildende und Schüler um ihnen bei ihrer informatischen Entwicklung zu helfen. 

\Clause{title=Verhältnis zur ``Gesellschaft für Informatik e. V.''}
\label{sec:verhältnisGI}

Die GI HSG stellt eine Hochschulgruppe nach dem Hochschulgruppenkonzept der ``Gesellschaft für Informatik e. V.'' dar. \\
Die GI HSG hat sich nach den Zielen und Werten der ``Gesellschaft für Informatik e. V.'' zu richten.

Die Postionen der SprecherIn und der stellvertretenden SprecherIn sind mit der vorsitzenden beziehungsweise stellvertretenden vorsitzenden Person nach \ref{sec:vorstand-posten}{} zu besetzen.

\Clause{title=Mitgliedschaft}

Ordentliche Mitglieder der Initiative können auf formlosen Antrag hin nur Studierende aller Fächer der Informatik und Wirtschaftsinformatik werden, die an der Universität Hagen immatrikuliert sind. Zusätzlich können auch Auszubildende und Schüler über einen formlosen Antrag hin angenommen werden. \\
Über Ausnahmen bei anderen Studienfächern mit Informatikanteil entscheidet der Vorstand.

Über den Mitgliedsantrag entscheidet der Vorstand. Eine Ablehnung ist schriftlich zu begründen

Die Mitgliedschaft ist nicht übertragbar, die Ausübung der Mitgliedschaftsrechte kann nur höchstpersönlich erfolgen. 

Eine Mitgliedschaft ist weder parteipolitisch noch konfessionell gebunden.

Die Mitgliedschaft ist unentgeltlich. Alle tätigen Mitglieder sind ehrenamtlich tätig.

\Clause{title=Mitgliedschaftsende}
\label{sec:mitgliedschaftsende}

Die Mitgliedschaft in der Initiative endet durch
\begin{enumerate}
    \item Exmatrikulation
    \item Austritt
    \item Ausschluss
\end{enumerate}
Ein Ausschluss ist schriftlich zu begründen und muss durch den Vorstand beschlossen werden.

\Clause{title=Beiträge}
\label{sec:beiträge}

Die Initiative erhebt keine Beiträge.


\Clause{title=Geschäftsjahr}
\label{sec:geschäftsjahr}

Geschäftsjahr ist das Kalenderjahr.

\Clause{title=Finanzierung}

Die Initiative finanziert sich aus den Mitteln der Studierendenschaft der Universität Hagen, Mitteln der Gesellschaft für Informatik, aus Spenden und aus sonstigen Einnahmen.

Zuwendungen Dritter dürfen nur angenommen werden, wenn sie nicht mit Auflagen verbunden sind, die den satzungsmäßigen Zwecken der Initiative zuwiderlaufen.


\Clause{title=Organe der Initiative}
\label{sec:organe}

Organe der Initiative sind:
\begin{enumerate}
    \item der Vorstand,
    \item die Mitgliederversammlung.
\end{enumerate}
Durch Beschluss der Mitgliederversammlung oder des Vorstands können Ausschüsse zur Wahrnehmung besonderer Aufgaben eingerichtet werden.


\Clause{title=Vorstand}
\label{sec:vorstand}


Der Vorstand besteht aus 
\begin{itemize}
    \item der Vorstandsvorsitzenden Person,
    \item der stellvertretenden Vorstandsvorsitzenden Person
    \item der Sekretär
\end{itemize}  \label{sec:vorstand-posten}
Im Vorstand müssen sich Mitglieder verschiedener Geschlechtsidentitäten befinden. Ausschlaggebend hierfür ist der Geschlechtseintrag in den Stammdaten des Studierendenservice zum Wahlzeitpunkt.

Vorstandsmitglieder müssen Mitglieder der Gesellschaft für Informatik sein.

Der Sekretär ist gleichzeitig Schatzmeister und für die Protokollführung zuständig.

Die Amtsperiode des Vorstands endet mit Ablauf des Wintersemesters oder der jederzeit möglichen Wahl eines neuen Vorstands durch Zweidrittel der anwesenden Mitglieder der Mitgliederversammlung. \label{sec:vorstand-VorzeitigeAbwahl}

Beschlüsse trifft der Vorstand mit einfacher Stimmenmehrheit. Der Vorstand ist beschlussfähig, wenn mehr als die Hälfte seiner stimmberechtigten Mitglieder anwesend sind.

Der Vorstand kann Mitglieder der Initiative als beratende Vorstandsmitglieder ernennen. Die Ernennung ist auf die Dauer der aktuellen Amtsperiode beschränkt.


\Clause{title=Geschäftsbereich des Vorstands}
\label{sec:geschäftsbereichVorstand}

Der Vorstand führt die Geschäfte der Initiative.

Der amtierende Vorstand trägt Sorge dafür, der AStA der Universität Hagen zur Kontaktaufnahme eine E-Mailadresse mitzuteilen, die er regelmäßig pflegt.


\Clause{title=Mitgliederversammlung}
\label{sec:mitgliederversammlung}

Die ordentliche Mitgliederversammlung wird durch den Vorstand einberufen. Sie findet mindestens einmal im Hochschuljahr und nicht während der vorlesungsfreien Zeit statt. Die ordentlichen Mitglieder der Initiative sind unter Bekanntgabe der Tagesordnung mindestens zwei Wochen vor dem Tag der Mitgliederversammlung per E-Mail einzuladen.

Der Vorstand kann im Interesse der Initiative eine außerordentliche Mitgliederversammlung einberufen. Er ist hierzu verpflichtet, wenn ein Fünftel der Mitglieder dies in Textform unter Angabe des Zwecks und der Gründe verlangt. In diesem Fall sind die Mitglieder unter Bekanntgabe der Tagesordnung mindestens zwei Wochen vor dem Tag der außerordentlichen Mitgliederversammlung in Textform einzuladen.


\Clause{title=Aufgaben der Mitgliederversammlung}
\label{sec:aufgabenMitgliederversammlung}

Die Angelegenheiten der Initiative werden, soweit sie nicht vom Vorstand zu erledigen sind, durch Beschlussfassung in der Mitgliederversammlung geregelt.
Die Mitgliederversammlung hat folgende Aufgaben:
\begin{enumerate}[label={(\arabic*)}]
    \item Wahl des Vorstands
    \item Vorzeitige Ab- und Neuwahl des Vorstands gemäß \ref{sec:vorstand-VorzeitigeAbwahl}
    \item Entlastung des Vorstands
    \item Beschlussfassung über
    \begin{enumerate}[label={\arabic*.}]
        \item  Die Einrichtung von Ausschüssen und die Festlegung ihrer Kompetenzen
        \item Satzungsänderungen
        \item Mitgliederausschluss
        \item Auflösung der Initiative.
    \end{enumerate}
\end{enumerate}


\Clause{title=Beschlussfassung der Mitgliederversammlung}

\label{sec:beschlussfassungMitgliederversammlung}

Die Mitgliederversammlung ist beschlussfähig, wenn sie ordnungsgemäß eingeladen wurde und mindestens 25\% der Mitglieder anwesend sind.

Jedes ordentliche Mitglied der Initiative ist antragsberechtigt. Jedes in der Mitgliederversammlung anwesende ordentliche Mitglied ist stimmberechtigt und hat eine Stimme. Beschlüsse werden mit einfacher Stimmenmehrheit gefasst, soweit gesetzliche Vorschriften oder die Satzung nichts anderes bestimmen.



\Clause{title=Niederschrift}
\label{sec:niederschrift}

Über alle Mitgliederversammlungen und Vorstandssitzungen ist eine Niederschrift anzufertigen. Die Protokolle müssen in geeigneter Form den Mitgliedern und allen Organen der Studierendenschaft durch den Vorstand zugänglich gemacht werden.


\Clause{title=Rechenschaftsbericht}
\label{sec:rechenschaftsbericht}

Der Vorstand dokumentiert die Verwendung studentischer Gelder durch die Initiative und hat die Aufgabe zum Ende des Kalenderjahres einen Rechenschaftsbericht anzufertigen, der bis zum 31. Januar beim Präsidium und beim Haushaltsausschuss des Studierendenparlaments, sowie beim Finanzreferat des AStA einzureichen ist.


\Clause{title=Satzungsänderung}
\label{sec:satzungsänderung}

Satzungsänderungen können nur mit der Mehrheit von zwei Dritteln der anwesenden stimmberechtigten Mitglieder einer Mitgliederversammlung beschlossen werden. Die vorgeschlagene Änderung ist als Tagesordnungspunkt bekannt zu geben und mit der Einladung an die ordentlichen Mitglieder zu versenden. Jede Änderung der Satzung ist dem Studierendenparlament unverzüglich in Textform  mitzuteilen.


\Clause{title=Auflösung der Initiative}
\label{sec:auflösung}

Die Initiative kann nur auf einer eigens dafür einzuberufenden Mitgliederversammlung durch Beschluss aufgelöst werden. Zur Auflösung ist die Mehrheit von drei Vierteln der anwesenden Mitglieder erforderlich. \label{sec:auflösung-beschluss}

Bei Auflösung der Initiative fällt das Vermögen aus Studentischen Geldern an den AStA der Universität Hagen.  Weiteres Vermögen geht an die ``Gesellschaft für Informatik e. V.''. Die Verwendung ist an den Zweck der Initiative gebunden. Genauere Einzelheiten hierzu beschließt die Mitgliederversammlung mit dem Beschluss nach \ref{sec:auflösung-beschluss}.


\Clause{title=Salvatorische Klausel}

Diese Satzung richtet sich nach der Satzung und den Richtlinien der Studierendenschaft der Universität Hagen, sowie interessierte Auszubildende und Schüler.

Sollten einzelne Bestimmungen dieser Satzung ganz oder teilweise ungültig sein oder werden, wird dadurch der Bestand der übrigen Satzung nicht berührt.

Unwirksame Bestimmungen sind durch gültige Bestimmungen zu ersetzen.



\end{contract}
