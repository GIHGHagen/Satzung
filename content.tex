\pagestyle{fancy}
\begin{contract}
    \Clause{title=Name und Sitz}
\label{sec:name}

Die studentische Vereinigung führt den Namen ``Gesellschaft für Informatik Hochschulgruppe Hagen'' (GI HSG).  Sie hat ihren Sitz in Hagen.


\Clause{title=Zweck der studentischen Vereinigung}
\label{sec:zweck}

Der satzungsmäßige Zweck der studentischen Vereinigung ist die Förderung des Fachbereichs Informatik an der Hochschule sowie die Unterstützung und Weiterentwicklung des Studiums, der Forschung und des Austauschs von Wissen im Bereich der Informatik. 
Die Hochschulgruppe strebt die Vernetzung von Studierenden, Lehrenden und Fachleuten der Informatik an, um den Dialog, die Kooperation und die Zusammenarbeit in diesem Fachbereich inneruniversitär als auch überregional zu fördern.
Auch richten wir uns an Auszubildende und Schüler um ihnen bei ihrer informatischen Entwicklung zu helfen. 

\Clause{title=Verhältnis zur ``Gesellschaft für Informatik e. V.''}
\label{sec:verhältnisGI}

Die GI HSG stellt eine Hochschulgruppe nach dem Hochschulgruppenkonzept der ``Gesellschaft für Informatik e. V.'' dar. \\
Die GI HSG hat sich nach den Zielen und Werten der ``Gesellschaft für Informatik e. V.'' zu richten.

Die Postionen der Sprecher:in und der stellvertretenden Sprecher:in sind mit der vorsitzenden beziehungsweise stellvertretenden vorsitzenden Person nach \ref{sec:vorstand-posten}{} zu besetzen.

\Clause{title=Mitgliedschaft}

Ordentliche Mitglieder der studentischen Vereinigung können auf formlosen Antrag hin nur Studierende aller Fächer der Informatik und Wirtschaftsinformatik werden, die an der Universität Hagen immatrikuliert sind. Zusätzlich können auch Auszubildende und Schüler über einen formlosen Antrag hin angenommen werden. \\
Über Ausnahmen bei anderen Studienfächern mit Informatikanteil entscheidet der Vorstand.

Über den Mitgliedsantrag entscheidet der Vorstand. Eine Ablehnung ist schriftlich zu begründen

Die Mitgliedschaft ist nicht übertragbar, die Ausübung der Mitgliedschaftsrechte kann nur höchstpersönlich erfolgen. 

Eine Mitgliedschaft ist weder parteipolitisch noch konfessionell gebunden.

Die Mitgliedschaft ist unentgeltlich. Alle tätigen Mitglieder sind ehrenamtlich tätig.

\Clause{title=Mitgliedschaftsende}
\label{sec:mitgliedschaftsende}

Die Mitgliedschaft in der studentischen Vereinigung endet durch
\begin{enumerate}
    \item Exmatrikulation
    \item Austritt
    \item Ausschluss
\end{enumerate}
Ein Ausschluss ist schriftlich zu begründen und muss durch den Vorstand oder durch die Mitgliederversammlung beschlossen werden.

\Clause{title=Beiträge}
\label{sec:beiträge}

Die studentische Vereinigung erhebt keine Beiträge.


\Clause{title=Geschäftsjahr}
\label{sec:geschäftsjahr}

Geschäftsjahr ist das Kalenderjahr.

\Clause{title=Finanzierung}

Die studentische Vereinigung finanziert sich aus den Mitteln der Studierendenschaft der Universität Hagen, Mitteln der Gesellschaft für Informatik, aus Spenden und aus sonstigen Einnahmen.

Zuwendungen Dritter dürfen nur angenommen werden, wenn sie nicht mit Auflagen verbunden sind, die den satzungsmäßigen Zwecken der studentischen Vereinigung zuwiderlaufen.


\Clause{title=Organe der studentischen Vereinigung}
\label{sec:organe}

Organe der studentischen Vereinigung sind:
\begin{enumerate}
    \item der Vorstand,
    \item die Mitgliederversammlung.
\end{enumerate}
Durch Beschluss der Mitgliederversammlung oder des Vorstands können Ausschüsse zur Wahrnehmung besonderer Aufgaben eingerichtet werden.


\Clause{title=Vorstand}
\label{sec:vorstand}


Der Vorstand besteht aus 
\begin{itemize}
    \item der vorsitzenden Person,
    \item der stellvertretenden Person
    \item der Sekretärsperson
\end{itemize}  \label{sec:vorstand-posten}
Im Vorstand müssen sich Mitglieder verschiedener Geschlechtsidentitäten befinden. Ausschlaggebend hierfür ist der Geschlechtseintrag in den Stammdaten des Studierendenservice zum Wahlzeitpunkt.

Mindestens der Vorsitzende und die Sekretärsperson müssen Mitglieder der Gesellschaft für Informatik sein.

Der oder die vorsitzende Person ist zuständig für Corporate Relations und Events. Sie muss zum Wahlzeitpunkt Mitglied der studentischen Vereinigung sein.

Die stellvertretende Person im Vorstand ist zuständig für Marketing und Projekte.

Der Sekretär:in ist gleichzeitig Schatzmeister:in und für die Protokollführung zuständig.

Die Amtsperiode des Vorstands endet mit Ablauf des Wintersemesters oder der jederzeit möglichen Wahl eines neuen Vorstands durch Zweidrittel der anwesenden Mitglieder der Mitgliederversammlung. \label{sec:vorstand-VorzeitigeAbwahl}

Beschlüsse trifft der Vorstand mit einfacher Stimmenmehrheit. Der Vorstand ist beschlussfähig, wenn mehr als die Hälfte seiner stimmberechtigten Mitglieder anwesend sind.

Der Vorstand kann Mitglieder der studentischen Vereinigung als beratende Vorstandsmitglieder ernennen. Die Ernennung ist auf die Dauer der aktuellen Amtsperiode beschränkt.

Vorstandsmitglieder nach Abs. 1 bleiben solange kommissarisch im Amt, bis ein neues Vorstandsmitglied gewählt wurde, welches das Amt übernimmt. Ist ein Vorstandsamt nach Abs. 1 nur kommissarisch besetzt, muss der Vorstand unverzüglich eine Mitgliederversammlung einberufen um das entsprechende Amt neu zu besetzen. Fristen nach \ref{sec:mitgliederversammlung} bleiben unberührt. Kommissarische Vorstandsmitglieder dürfen nur absolut notwendige Tätigkeiten durchführen. Hierzu zählt insbesondere das Einberufen einer Mitgliederversammlung nach \ref{sec:mitgliederversammlung}.

Vorstandsmitglieder können von ihrem Vorstandsamt zurücktreten. Der Rücktritt ist in Textform dem Vorstand oder der vorsitzenden Person zur Niederschrift anzuzeigen. Ist das Amt nicht nach Abs. 1 oder den gesetzlichen Bestimmungen zu besetzen endet die Amtszeit des Vorstandsmitglieds mit dem Rücktritt.

\Clause{title=Geschäftsbereich des Vorstands}
\label{sec:geschäftsbereichVorstand}

Der Vorstand führt die Geschäfte der studentischen Vereinigung.

Der amtierende Vorstand trägt Sorge dafür, der AStA der Universität Hagen zur Kontaktaufnahme eine E-Mailadresse mitzuteilen, die er regelmäßig pflegt.


\Clause{title=Mitgliederversammlung}
\label{sec:mitgliederversammlung}

Die ordentliche Mitgliederversammlung wird durch den Vorstand einberufen. Sie findet mindestens einmal im Hochschuljahr und nicht während der vorlesungsfreien Zeit statt. 

Der Vorstand kann im Interesse der studentischen Vereinigung eine außerordentliche Mitgliederversammlung einberufen. Er ist hierzu verpflichtet, wenn ein Fünftel der Mitglieder dies in Textform unter Angabe des Zwecks und der Gründe verlangt. 

Die ordentlichen Mitglieder der studentischen Vereinigung sind unter Bekanntgabe der Tagesordnung mindestens zwei Wochen vor dem Tag der Mitgliederversammlung per E-Mail einzuladen.

\Clause{title=Aufgaben der Mitgliederversammlung}
\label{sec:aufgabenMitgliederversammlung}

Die Angelegenheiten der studentischen Vereinigung werden, soweit sie nicht vom Vorstand zu erledigen sind, durch Beschlussfassung in der Mitgliederversammlung geregelt.
Die Mitgliederversammlung hat folgende Aufgaben:
\begin{enumerate}[label={(\arabic*)}]
    \item Wahl des Vorstands
    \item Vorzeitige Ab- und Neuwahl des Vorstands gemäß \ref{sec:vorstand-VorzeitigeAbwahl}
    \item Entlastung des Vorstands
    \item Beschlussfassung über
    \begin{enumerate}[label={\arabic*.}]
        \item  Die Einrichtung von Ausschüssen und die Festlegung ihrer Kompetenzen
        \item Satzungsänderungen
        \item Mitgliederausschluss
        \item Auflösung der studentischen Vereinigung.
    \end{enumerate}
\end{enumerate}


\Clause{title=Beschlussfassung der Mitgliederversammlung}

\label{sec:beschlussfassungMitgliederversammlung}

Die Mitgliederversammlung ist beschlussfähig, wenn sie ordnungsgemäß eingeladen wurde und mindestens 25\% der Mitglieder anwesend sind.

Jedes ordentliche Mitglied der studentischen Vereinigung ist antragsberechtigt. Jedes in der Mitgliederversammlung anwesende ordentliche Mitglied ist stimmberechtigt und hat eine Stimme. Beschlüsse werden mit einfacher Stimmenmehrheit gefasst, soweit gesetzliche Vorschriften oder die Satzung nichts anderes bestimmen.



\Clause{title=Niederschrift}
\label{sec:niederschrift}

Über alle Mitgliederversammlungen und Vorstandssitzungen ist eine Niederschrift anzufertigen. Die Protokolle müssen in geeigneter Form den Mitgliedern und allen Organen der Studierendenschaft durch den Vorstand zugänglich gemacht werden.


\Clause{title=Rechenschaftsbericht}
\label{sec:rechenschaftsbericht}

Der Vorstand dokumentiert die Verwendung studentischer Gelder durch die studentische Vereinigung und hat die Aufgabe zum Ende des Kalenderjahres einen Rechenschaftsbericht anzufertigen, der bis zum 31. Januar beim Präsidium und beim Haushaltsausschuss des Studierendenparlaments, sowie beim Finanzreferat des AStA einzureichen ist.


\Clause{title=Satzungsänderung}
\label{sec:satzungsänderung}

Satzungsänderungen können nur mit der Mehrheit von zwei Dritteln der anwesenden stimmberechtigten Mitglieder einer Mitgliederversammlung beschlossen werden. Die vorgeschlagene Änderung ist als Tagesordnungspunkt bekannt zu geben und mit der Einladung an die ordentlichen Mitglieder zu versenden. Jede Änderung der Satzung ist dem Studierendenparlament unverzüglich in Textform  mitzuteilen.


\Clause{title=Auflösung der studentischen Vereinigung}
\label{sec:auflösung}

Die studentische Vereinigung kann nur auf einer eigens dafür einzuberufenden Mitgliederversammlung durch Beschluss aufgelöst werden. Zur Auflösung ist die Mehrheit von drei Vierteln der anwesenden Mitglieder erforderlich. \label{sec:auflösung-beschluss}

Bei Auflösung der studentischen Vereinigung fällt das Vermögen aus Studentischen Geldern an den AStA der Universität Hagen.  Weiteres Vermögen geht an die ``Gesellschaft für Informatik e. V.''. Die Verwendung ist an den Zweck der studentischen Vereinigung gebunden. Genauere Einzelheiten hierzu beschließt die Mitgliederversammlung mit dem Beschluss nach \ref{sec:auflösung-beschluss}.


\Clause{title=Salvatorische Klausel}

Diese Satzung richtet sich nach der Satzung und den Richtlinien der Studierendenschaft der Universität Hagen, sowie interessierte Auszubildende und Schüler.

Sollten einzelne Bestimmungen dieser Satzung ganz oder teilweise ungültig sein oder werden, wird dadurch der Bestand der übrigen Satzung nicht berührt.

Unwirksame Bestimmungen sind durch gültige Bestimmungen zu ersetzen.



\end{contract}
